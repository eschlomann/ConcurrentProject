\documentclass[FinalReport.tex]{subfiles}
\begin{document}

\bigskip

\section*{\textsc{\Large Techniques}}

	To have a system in which shared variables are not manipulated at the same time, we develop ways that threads will wait on others until it is their turn to edit the variable.  
	
	\begin{itemize}
		\item \textsc{ \bf MCH \& CLH} \\
			Some locks, such as MCS and CLH, use a queue to place in order the threads that wish to enter the critical section.  This method is a first come first served system.  One thread (T1) will "announce" that it wishes to enter the critical section, and if before, another thread (T2) "announced" it also wants to enter the critical section.  The thread T1 will have to wait until T2 has completed its tasks in the critical section.
			
		\item \textsc{ \bf Bakery}\\
		The method used by the Bakery Algorithm is also First come first served.  This method uses a ticketing system.  This system has each thread that wants to enter the critical section take a number.  These number are increasing in size.  The thread with the next lowest number will then enter the critical section as the other threads wait until their turn.

		\item \textsc{ \bf Eisenberg \& McGuire}\\
		This Eisenberg \& McGuire Algorithm cycles through waiting threads to establish a bound.  The priority of each thread is rotated around this "cycle".  The thread that is in the critical section or most recently in the critical section has the highest priority.  The threads waiting in the "clockwise" direction have decreasing priority levels from the highest priority thread.  There are three different states, (Don't want the critical section, Want to go into the critical section, and In the critical section) to indicate intention. 
		
		\item \textsc{ \bf Szyma\`{n}ski's}\\
		Szyma\`{n}ski's Algorithm is modeled around a waiting room with an entry and an exit. Many threads may enter the waiting room, but only one may enter the critical section at a time in this model. The model focuses around positions stored in an array. A "1" indicates that a thread is outside the waiting room and is interested. A "3" indicates that a thread is standing in the doorway. A "2" says that the thread is standing to the side of the doorway waiting for others to enter. A "4" means that the door is closed. The process to get into the critical section is as follows. When the thread calls lock it sets its flag to a "1", it then waits for all flags to be a "0", "1", or "2". Then it sets its flag to a "3" and stands in the doorway. If their are other flags interested, it steps aside and sets its value to a "2" and waits for another thread to enter. Once it enters, the thread sets its flag to a "4". This means that it is in the waiting room and the door is closed. The thread then awaits everyone of a lower thread ID to finish before it enters the critical section. In high contention cases, Szyma\`{n}ski's Algorithm heavily favors lower thread IDs and is in no way fair.
		
	\end{itemize}
	
	
	
	
\end{document}
