\documentclass[FinalReport.tex]{subfiles}
\begin{document}

\bigskip

\section*{\textsc{\Large Discussion}}

The primary purpose of this project is to form understanding of when certain types of locks are successful. This can be broken down in to several different areas such as: number of threads, speed of system, and amount of contention. The results shed some light on when certain systems work and when they should be avoided. The general benchmark that all lock types will be compared to is java's Reentrant Lock written by Doug Lee, as this is a widely used system.

The first scenario is the most simple one, all of the locks running a simple model where each thread attempts to increment a counter 1000 times with 1 - 8 threads. This is the first set of images in the Results section. Up to 7 threads, the ranking of lock types between the faster and slower system are close to consistent one notable differences. First the Fair Reentrant lock outperformed others on the slower system where it was near slowest on the fast system. The fair Reentrant lock was expected to perform the slowest, as it provides the greatest amount of stability and fairness in a system, but surprisingly at a every number of threads the CLH lock showed very poorly, quickly scaling to becoming worse than the Fair Reentrant lock. This performance is likely much worse in a high contention scenario because of the memory layout of the system. 

As discussed in the motivating examples section the java ReentrantLock accepts a boolean, whether or not the lock will be fair. This condition makes a large difference in performance. Other locks that also have fairness such as MCS, CLH, and Bakery were close to or outperformed the Fair Reentrant lock in every high contention test. The unfair locks: normal Reentrant, Eisenberg McGuire, and Szymanski had differing results. The Eisenberg McGuire Algorithm and the normal java reentrant lock were near indecipherable in performance as the best lock between 1 and 8 threads. This is due to their optimization. The Reentrant lock cycles quickly to find a new thread once one is out of the critical section, and the Eisenberg McGuire algorithm is the first solution to n thread contention solved in n-1 turns. 

The second scenario is similar to the first but run at higher thread counts, the slower system processed these results too slowly to garner any meaningful data. What may not be apparent from simply looking at the graph is that the java lock as well as the java fair lock are sitting at the very bottom of the graph as there performance is far superior to any of the other locks in this case. Of the remaining locks, the MCS lock and the Eisenberg McGuire algorithm are the best performers while the Szymanski algorithm slips into last place. The locks implemented by java perform better as they interact more closely with the hardware implementation, rather than focusing performance on optimization with low thread counts they perform well at any thread count making them useful on any system. An important note is that with the exception of the java locks, once 9 threads is reached performance decreases by about a factor of 10 and then scales near linearly again. This is due to the way that hyper-threading works on a 4 core system. After 8 threads hyper threading becomes a far more difficult task that will take much longer. 

The third scenario is the most straight forward. Testing with low contention should give each lock its optimal case. This shows in the results, the execution scales fairly linearly with time and the locks are all within around 1 millisecond of each other in every case.

Overall these tests have shown that if there are a small number of threads, high contention, and fairness needs to be preserved, there are better options than the java Fair locks, but otherwise, especially at high thread counts, the java Reentrant lock outperforms other locking algorithms.

\end{document}
